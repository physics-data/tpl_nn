\documentclass{article}
\usepackage{amsmath,amsfonts,amsthm,amssymb}
\usepackage{tikz}

\topmargin=-0.45in
\evensidemargin=0in
\oddsidemargin=0in
\textwidth=6.5in
\textheight=9.0in
\headsep=0.25in

\title{\textbf{Big Data in Experimental Physics: NN}}
\author{TAs}

\begin{document}
\maketitle

\section{Back propagation}

As \texttt{README.md} has explained, the computation process is

\begin{align*}
  v^{(1)}&=W^{(1)}v^{(0)}\\
  v^{(2)}&=\text{ReLu}(v^{(1)})\\
  v^{(3)}&=W^{(2)}v^{(2)}
\end{align*}

The loss is computed in the following way
\begin{align*}
  E&=\frac{1}{2}\sum_{i}(v^{(3)}_i-y_i)^2
\end{align*}

\newcommand{\pfrac}[2]{\frac{\partial #1}{\partial #2}}

To minimize the loss, we can use derivative to update the weights as $E$ is a function of all weights. The gradient for all the weights can be obtained in a way similar to the following process.
\begin{align*}
  \pfrac{E}{v^{(3)_i}}&=v^{(3)}_i-y_i\\
  \pfrac{E}{W^{(2)}_{i,j}}\
  &=\pfrac{E}{v^{(3)_i}}\pfrac{v^{(3)}_i}{W^{(2)}_{i,j}}\\
  &=\pfrac{E}{v^{(3)_i}}v^{(2)}_j\\
  \pfrac{E}{v^{(2)}_j}
  &=\sum_i\pfrac{E}{v^{(3)}_i}\pfrac{v^{(3)}_i}{v^{(2)}_{j}}\\
  &=\sum_i\pfrac{E}{v^{(3)}_i}W^{(2)}_{i,j}\\
\end{align*}

As for the ReLU function,
\begin{align*}
  \pfrac{v^{(2)}_i}{v^{(1)}_i}
  &=\left\{
    \begin{array}{ll}
      1 & v^{(1)}_i>0\\
      0 & v^{(1)}_i\leq 0
    \end{array}
    \right.
\end{align*}

The second half of the problem is to compute
\begin{align*}
  \text{argmax}_{i,j}\left|\pfrac{E}{W^{(1)}_{i,j}}\right|
\end{align*}


\end{document}
